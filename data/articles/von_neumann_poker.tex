\documentclass{article}
\usepackage[margin=1.25in]{geometry}
\usepackage{amsmath,amssymb,graphicx}
\usepackage[colorlinks=true, linkcolor=black, urlcolor=black]{hyperref}
% \usepackage{fancyhdr}
% \pagestyle{fancy}
% \fancyhf{}

\title{Bayes–Nash Equilibrium in a One-Street No-Limit Von Neumann Poker Model}
\author{Grant Stenger}
\date{February 2025}

\begin{document}

\maketitle

\section{Introduction and Setup}

We consider a simplified one-street poker game between two players, Alice and Bob. Each player's private card strength is drawn uniformly from the interval $[0,1]$. Let:

\begin{aligned}
\alpha \sim \text{Uniform}(0,1) \; \text{(Alice's private card)} \\
\beta \sim \text{Uniform}(0,1) \; \text{(Bob's private card)}
\end{aligned}

We assume that $\alpha$ and $\beta$ are independent.

At the start of the game, each player antes $a$ chips, forming a pot of size $2a$.  We consider a single betting round:

\begin{enumerate}
    \item Alice observes her private card $\alpha$ and may choose a bet size $b$ such that $0 \leq b \leq S_A$ (where $S_A$ is her remaining stack or maximum allowable bet).
    \item Bob observes his private card $\beta$ (but not $\alpha$) and Alice's bet $b$. Then Bob must either \emph{call} (put in $b$ chips) or \emph{fold}:
    \begin{itemize}
        \item If Bob folds, Alice immediately wins the pot $2a + b$.  Since she contributed $a + b$ total, her net profit is effectively $+a$ (Bob's ante).
        \item If Bob calls, then each has contributed $a + b$ total.  The final pot is $2a + 2b$.  
        \begin{itemize}
            \item If $\alpha > \beta$, Alice wins the pot $2a + 2b$, netting $+(a + b)$ from the hand.
            \item If $\beta > \alpha$, Bob wins $2a + 2b$, netting $+(a + b)$, and Alice loses $a + b$.
        \end{itemize}
    \end{itemize}
\end{enumerate}

We assume \emph{risk neutrality} for both players: they aim to maximize the expected monetary value of the chips they gain in this single hand.\footnote{We assume both players have high enough net worths relative to the value of this game such that even though their overall utility functions are likely to be convex in wealth, in this game they are approximately risk-neutral, meaning they maximize linear utility (the direct expected monetary value of this game).} Our goal is to find a \emph{Bayesian Nash Equilibrium} (BNE) in mixed strategies:

\begin{align*}
\Pi_{A}^* : \text{Alice's equilibrium betting strategy as a function of }\alpha \\
\Pi_{B}^* : \text{Bob's equilibrium calling/folding strategy as a function of }\beta
\end{align*}

\section{Bob's Decision: Call or Fold}

To solve this game, we start at the end and use backwards induction at each critical decision point. We begin by analyzing Bob's decision after seeing Alice's bet $b$ and his private card $\beta$. Bob compares:
\[
\text{EV}(\text{Call}\mid b, \beta), \quad \text{EV}(\text{Fold}\mid b, \beta).
\]

\subsection{Folding}

If Bob folds, he abandons the pot and loses his ante $a$.  Because the ante is a sunk cost, one can set
\[
\text{EV}(\text{Fold}) = 0
\]
relative to Bob's situation at that decision point (folding yields no additional gain or loss).

\subsection{Calling}

If Bob calls, he invests $b$ more chips.  The pot becomes $2a + 2b$.  If Bob has $\beta > \alpha$, he wins the pot for a net $2a + b$ (considering the ante as sunk).  If $\beta < \alpha$, he loses his call of $b$.

Denote
\[
p \;=\; \Pr(\beta > \alpha \mid \text{Alice bet } b, \text{ Bob's card } \beta ).
\]
Bob's expected value from calling is:
\[
\text{EV}(\text{Call}) = p\,(2a + b) + (1-p)\,(-b).
\]
He calls if and only if $\text{EV}(\text{Call}) \ge \text{EV}(\text{Fold}) = 0$.  The \emph{indifference condition} is:
\[
p\,(a + b) - (1-p)\,b = 0 
\;\;\Longrightarrow\;\;
p = \frac{b}{\,2a + 2b\,}.
\]
Hence Bob will call exactly when 
\[
\Pr(\beta > \alpha \mid b, \beta) \;\ge\; \frac{b}{2a + 2b}.
\]

\section{Alice's Decision: Bet or Not}

Next, consider Alice's perspective.  Given her card $\alpha$, she must decide whether to bet some $b>0$ or to check ($b=0$).  We will see that in equilibrium, Alice typically bets only with her best (value-bet) \emph{and} worst (bluff) hands, and checks with middling hands.

\subsection{If Alice Checks ($b=0$)}

If Alice checks, no additional money goes into the pot.  The game proceeds directly to showdown with the pot size at $2a$.  

\[
\begin{cases}
\alpha > \beta \;\;\Longrightarrow\;\; \text{Alice wins }2a \text{, net }+a.\\
\alpha < \beta \;\;\Longrightarrow\;\; \text{Alice loses, net }-a.
\end{cases}
\]


Since $\beta$ is uniform on $[0,1]$ and $\alpha$ is given, $\Pr(\beta < \alpha \, | \, \alpha) = \alpha$ and $\Pr(\beta > \alpha \, | \, \alpha) = 1-\alpha$.  Thus
\[
\text{EV}(\text{Check}\mid \alpha)
= \alpha \cdot a + (1-\alpha)\cdot (-a) 
= a\,(2\alpha - 1).
\]


\subsection{If Alice Bets $b > 0$}

If Alice bets $b$, Bob will respond by calling if his posterior probability of having $\beta > \alpha$, conditional on observing $b$, is at least $\frac{b}{2a + 2b}$, and fold otherwise. In equilibrium, Bob's strategy simplifies to a cut-off threshold $c$: Bob calls if $\beta > c$ and folds if $\beta < c$, for some $c \in [0,1]$. The exact $c$ will be derived by ensuring Bob is indifferent at $\beta = c$.

From Alice's viewpoint, if Bob folds, she \emph{immediately} wins $(2a + b)$, but note that she put up $a+b$, so her net gain is $+a$ (Bob's ante). If Bob calls, then:
\[
\begin{cases}
\alpha > \beta \implies \text{Alice's payoff} = +(a + b),\\
\alpha < \beta \implies \text{Alice's payoff} = -(a + b).
\end{cases}
\]

Thus Alice's expected payoff from betting, given that Bob's calling threshold is $c$, is:
\[
\begin{aligned}
\text{EV}(\text{Bet} \mid \alpha) 
=& \underbrace{\Pr(\beta < c)}_{\text{Bob folds}} \times (+a)
\;+\;
\underbrace{\Pr(\beta \ge c)}_{\text{Bob calls}} 
\Big[\Pr(\alpha > \beta \mid \beta \ge c)\cdot (a+b) + \Pr(\alpha < \beta \mid \beta \ge c)\cdot (-(a+b))\Big].
\end{aligned}
\]
Since $\beta$ is uniform, $\Pr(\beta < c) = c$ and $\Pr(\beta \ge c) = 1-c$. We must condition on $\beta \ge c$ to find $\Pr(\alpha > \beta \mid \beta \ge c)$. 

\paragraph{Case 1: $\alpha \le c$.} 
If $\alpha \le c$, then if Bob calls, $\beta \ge c \ge \alpha$, so $\alpha$ can never win the showdown. Hence
\[
\text{EV}(\text{Bet} \mid \alpha \le c) 
= c \cdot a + (1-c)\big(0 \cdot (a+b) + 1\cdot (-(a+b))\big)
= c\,a - (1-c)(a+b).
\]

\paragraph{Case 2: $\alpha \ge c$.} 
Then if Bob calls, $\beta \ge c$, but there is still a chance $\beta < \alpha$ within $[c,1]$. Specifically, $\beta$ is uniform on $[c,1]$, so $\Pr(\beta < \alpha \mid \beta\ge c) = \frac{\alpha - c}{1 - c}$, provided $\alpha > c$. Thus
\[
\begin{aligned}
\text{EV}(\text{Bet} \mid \alpha \ge c)
= 
& c\,a 
\;+\; (1-c) \Big[\frac{\alpha - c}{1-c}\cdot (a+b) - \frac{1 - \alpha}{1-c}\cdot (a+b)\Big]
\\
=& c\,a + \big((\alpha - c) - (1 - \alpha)\big) (a+b)
\\
=& c\,a + (2\alpha - 1 - c)(a+b).
\end{aligned}
\]

To summarize, for a given $c$:
\[
\text{EV}(\text{Bet} \mid \alpha) 
= \begin{cases}
c\,a - (1-c)(a+b), & \text{if } \alpha < c,\\[6pt]
c\,a + (2\alpha - 1 - c)(a+b), & \text{if } \alpha \ge c.
\end{cases}
\]








\section{Threshold Strategies and Indifference Conditions}

It is known (both from heuristic and classic game-theoretic results) that the unique Bayesian Nash Equilibrium in this model has a \emph{threshold form} for both players:

\[
\Pi_B^*(\beta)=
\begin{cases}
\text{call}, & \beta\ge c^*,\\
\text{fold}, & \beta<c^*,
\end{cases}
\qquad
\Pi_A^*(\alpha)=
\begin{cases}
\text{bet}, & \alpha\le a^*\;\text{or}\;\alpha\ge b^*,\\
\text{check}, & a^*<\alpha<b^*.
\end{cases}
\]

Intuitively, Alice bluffs with the worst hands ($\alpha\le a^*$) to exploit Bob's folds, value-bets with strong hands ($\alpha\ge b^*$) to extract calls from Bob, and checks middling hands. Bob calls with sufficiently strong holdings $\beta\ge c^*$ and folds the weaker ones.

\subsection{Bob's Indifference at $\beta = c^*$}

When Bob holds exactly $\beta = c^*$, he should be \emph{indifferent} between calling and folding.  That is,
\[
\Pr(\beta > \alpha \,\mid \text{Alice bet}, \beta=c^*) 
= \frac{b}{\,2a + 2b\,}.
\]
Because Alice bets only if $\alpha \le a^*$ or $\alpha \ge b^*$, we must compute $\Pr(\alpha < c^*\mid \text{Alice bet})$ under the assumption $0 \le a^* < c^* < b^* \le 1$.  In that situation:
\[
\Pr(\alpha < c^*\mid \alpha \in [0,a^*]\cup[b^*,1])
= \frac{\Pr(\alpha \in [0,a^*])}{\,\Pr(\alpha \in [0,a^*]\cup[b^*,1])\,}
= \frac{a^*}{\,a^* + (1-b^*)\,}.
\]
If $\beta=c^*\in(a^*,b^*)$, then Bob's chance of beating Alice upon seeing a bet is exactly the fraction of Alice's betting range that lies below $c^*$.  So Bob's indifference condition becomes
\begin{equation}
\frac{a^*}{\,a^* + (1-b^*)\,} 
= \frac{b}{\,2a + 2b\,}.
\label{Eq:BobIndiff}
\end{equation}

\subsection{Alice's Indifference at $\alpha = b^*$ (Value-Bet Boundary)}

At $\alpha=b^*$, Alice should be indifferent between checking and betting:
\[
\text{EV}(\text{Check}\mid b^*) = \text{EV}(\text{Bet}\mid b^*).
\]
Since $\text{EV}(\text{Check}\mid b^*) = a(2b^* -1)$, and
\[
\text{EV}(\text{Bet}\mid b^*)
= c^*\,a 
+ (1-c^*)\biggl[\frac{b^* - c^*}{1-c^*}(a+b) + \frac{1-b^*}{1-c^*}(-(a+b))\biggr]
= c^*\,a + (2b^* -1 - c^*)\,(a+b),
\]
setting them equal yields
\[
a(2b^*-1) \;=\; c^*\,a + (2b^* -1 - c^*)(a+b).
\]
A rearrangement shows that this forces
\begin{equation}
2b^* -1 - c^* = 0 \quad\Longrightarrow\quad c^*=2b^*-1.
\label{Eq:ValBound}
\end{equation}

\subsection{Alice's Indifference at $\alpha = a^*$ (Bluff Boundary)}

At $\alpha = a^*$, Alice is indifferent between checking, which yields
\[
\text{EV}(\text{Check}\mid a^*) = a(2a^*-1),
\]
and betting, which yields
\[
\text{EV}(\text{Bet}\mid a^*) 
= c^*\,a + (1-c^*)\bigl[\underbrace{\Pr(\alpha>\beta \mid \beta\ge c^*)}_{0\text{ if }a^*<c^*}(a+b) - (1-\Pr(\alpha>\beta))(a+b)\bigr].
\]
Since $a^*<c^*$, if Bob calls (i.e.\ $\beta\ge c^*$), then $\beta \ge c^* > a^* = \alpha$, so Alice always loses if called.  Hence $\Pr(\alpha>\beta)=0$ in the call case:
\[
\text{EV}(\text{Bet}\mid a^*) = c^*\,a - (1-c^*)(a+b).
\]
Equating $\text{EV}(\text{Check}) = \text{EV}(\text{Bet})$:
\[
a(2a^* -1) = c^*\,a - (1-c^*)(a+b).
\]
Rearranging,
\[
a(2a^* -1) - a\,c^* = -(1-c^*)(a+b),
\]
\[
a\bigl(2a^* -1 - c^*\bigr) = -(1-c^*)(a+b),
\]
\begin{equation}
    a\bigl(c^* +1 -2a^*\bigr) = (1-c^*)(a+b). \label{Eq:BluffBound}
\end{equation}


\section{Solving the System for \texorpdfstring{$(a^*,b^*,c^*)$}{(a*, b*, c*)}}
\label{sec:SolvingThresholds}

We now \emph{explicitly} solve for the three unknown thresholds $(a^*, b^*, c^*)$ in terms of $(a,b)$, i.e.\ the ante $a$ and the bet size $b$.  The system to solve is:

\begin{enumerate}
\item \emph{Value-bet boundary:} 
\[
c^* = 2b^* -1,
\quad
\text{(from \eqref{Eq:ValBound})}
\]

\item \emph{Bob's indifference:}
\[
\frac{a^*}{\,a^* + (1-b^*)\,} = \frac{b}{\,2a + 2b\,},
\quad
\text{(from \eqref{Eq:BobIndiff})}
\]

\item \emph{Bluff boundary:} 
\[
\text{At } \alpha=a^*,\; a(2a^*-1) = c^*\,a - (1-c^*)(a+b).
\quad
\text{(from \eqref{Eq:BluffBound})}
\]
% After algebraic rearrangement and substituting $c^*=2b^*-1$, one can solve for $a^*$ in terms of $(b^*,a,b)$.
\end{enumerate}

\subsection{Algebraic Solving}

\paragraph{(i) From Bob's indifference (\ref{Eq:BobIndiff}).}
We rewrite it as
\[
a^*(2a+2b) = b\bigl[a^* + 1 - b^*\bigr].
\]
Rearrange to isolate $a^*$:
\[
a^*(2a + 2b) - b\,a^* = b(1 - b^*),
\]
\[
a^*(2a + 2b - b) = b(1 - b^*),
\]
\[
a^*(2a + b) = b(1 - b^*),
\quad\Longrightarrow\quad
a^* = \frac{b\,(1 - b^*)}{\,2a + b\,}.
\label{A}
\tag{A}
\]

\paragraph{(ii) From the bluff boundary.}
We have
\[
a(2a^*-1) = c^*\,a - (1-c^*)(a+b).
\]
Since from \eqref{Eq:BobIndiff}, $c^*=2b^*-1$, we set 
\[
1 - c^* = 1 - (2b^* -1) = 2 - 2b^* = 2(1 - b^*).
\]
Then
\[
a(2a^* -1) 
= (2b^*-1)\,a \;-\; 2(1-b^*)(a+b).
\]
Solve directly for $a^*$.  Bring all terms in $a^*$ to one side:
% \[
% a(2a^* -1) = a(2b^* -1) - 2(1-b^*)(a+b).
% \]
\[
2a^* -1 = (2b^* -1) \;-\; \frac{2(1-b^*)(a+b)}{a}.
\]
\[
2a^* = 2b^* \;-\; \frac{2(1-b^*)(a+b)}{a}.
\]
Hence
\begin{equation}
    a^* = b^* - \frac{(1-b^*)(a+b)}{\,a\,}.
    \tag{B}
    \label{B}
\end{equation}



\paragraph{(iii) Combine (A) and (B).}
From \eqref{A}:
\[
a^* = \frac{ b(1 - b^*) }{\,2a + b\,}.
\]
From \eqref{B}:
\[
a^* = b^* \;-\; (1-b^*)\,\frac{a+b}{\,a\,}.
\]
Set these two expressions equal:
\[
\frac{b(1 - b^*)}{2a + b}
\;=\;
b^* \;-\; (1-b^*)\,\frac{a+b}{a}.
\]
Multiply through by $a(2a+b)$ to clear denominators:
\[
a(2a+b)\,\frac{b(1-b^*)}{2a+b}
\;=\;
a(2a+b)\Bigl[b^* - (1-b^*)\,\frac{a+b}{a}\Bigr].
\]
The left side simplifies to $a\,b\,(1-b^*)$.  The right side expands:
\[
a(2a+b)\,b^*
\;-\;
a(2a+b)\,(1-b^*)\frac{a+b}{a}.
\]
Hence
\[
a\,b\,(1-b^*)
=
a(2a+b)\,b^*
\;-\;
(2a+b)(1-b^*)(a+b).
\]
Move everything to one side:
\[
a(2a+b)\,b^*
\;-\;
(2a+b)(1-b^*)(a+b)
\;-\;
a\,b\,(1-b^*)
= 0.
\]



\paragraph{(iv) Solve for \boldmath{$b^*$}.}
Let us expand carefully.  Consider the three big terms:

\begin{itemize}
\item $a(2a+b)\,b^* = 2a^2\,b^* + a b\,b^*$.
\item $-(2a+b)\,(1-b^*)\,(a+b)\;\;=\;\;-(2a+b)(a+b) \;+\;(2a+b)(a+b)\,b^*$.
\item $-\,a\,b\,(1-b^*) \;=\; -\,a b \;+\; a b\,b^*$.
\end{itemize}
Combine like terms, carefully grouping those that contain $b^*$ versus those that do not:

\[
\underbrace{
2a^2\,b^*
\;+\;
a b\,b^*
\;+\;
(2a+b)(a+b)\,b^*
\;+\;
a b\,b^*
}_{\text{$b^*$-terms}}
\quad+\quad
\underbrace{
-\,(2a+b)(a+b)
\;-\;
a b
}_{\text{non-$b^*$ terms}}
\;=\;0.
\]
Factor out $b^*$ and simplify,
\[
b^*\Bigl[\,
2a^2 \;+\; 2ab \;+\; 2a^2 \;+\; 3ab \;+\; b^2
\Bigr]
\;-\;
\Bigl[
(2a+b)(a+b) \;+\; a b
\Bigr]
\;=\;0,
\]
\[
b^*\bigl[\,4a^2 \;+\;5ab \;+\; b^2\bigr]
\;-\;
\bigl[\,2a^2 \;+\;3ab \;+\; b^2 \;+\; a b\bigr]
\;=\;0,
\]
\[
b^*\bigl[4a^2 +5ab + b^2\bigr]
\;=\;
2a^2 +4ab + b^2.
\]
Therefore, 
\[
\boxed{
b^*
= \frac{\,2a^2 \;+\;4ab \;+\; b^2\,}{\,4a^2 \;+\;5ab \;+\; b^2\,}.
}
\]

\paragraph{(v) Obtain \boldmath{$c^* = 2b^*-1$}.}
Substitute the expression for $b^*$:
\[
c^*
= 2 \cdot \frac{\,2a^2 +4ab + b^2\,}{\,4a^2 +5ab + b^2\,}
\;-\;
1
= \frac{\,2(2a^2 +4ab + b^2)\, -\, (4a^2 +5ab + b^2)\,}{\,4a^2 +5ab + b^2\,}.
\]
The numerator is $\,2(2a^2 +4ab + b^2) = 4a^2 +8ab +2b^2\,$ minus $\,4a^2 +5ab + b^2$, yielding 
\[
(4a^2 +8ab +2b^2) \;-\; (4a^2 +5ab + b^2)
= 3ab + b^2.
\]
Hence,
\[
\boxed{
c^*
= \frac{\,3ab + b^2\,}{\,4a^2 \;+\;5ab \;+\; b^2\,}.
}
\]


\paragraph{(vi) Finally, solve for \boldmath{$a^*$}.}
From \eqref{A},
\[
a^*
= \frac{\,b\,(1-b^*)\,}{\,2a + b\,}.
\]
We already have
\[
b^*
= \frac{2a^2 +4ab + b^2}{4a^2 +5ab + b^2}
\;\;\Longrightarrow\;\;
1 - b^*
= 1
- \frac{2a^2 +4ab + b^2}{4a^2 +5ab + b^2}
= \frac{\,2a^2 + ab\,}{\,4a^2 +5ab + b^2\,}.
\]
Therefore
\[
b\,(1-b^*)
= b \cdot \frac{\,2a^2 + ab\,}{\,4a^2 +5ab + b^2\,}
= \frac{\,b\,(2a^2 + ab)\,}{\,4a^2 +5ab + b^2\,}.
\]
So
\[
a^*
= \frac{\,b(2a^2 + ab)\,}{\,(2a+b)\,\bigl(4a^2 +5ab + b^2\bigr)}.
\]
But notice $\,2a^2 + ab = a(2a + b)\,$, which cancels with the $(2a+b)$ in the denominator.  Thus
\[
\boxed{
a^*
= \frac{\,a\,b\,}{\,4a^2 \;+\;5ab \;+\; b^2\,}.
}
\]


\subsection{Final Equilibrium Thresholds}

We've found that there is a unique solution that satisfies our constraints:
\begin{itemize}
    \item \emph{Bluff range:} Alice bets when $\alpha \le a^*$.
    \item \emph{Value range:} Alice bets when $\alpha \ge b^*$.
    \item \emph{Check range:} $a^* < \alpha < b^*$.
    \item Bob calls if $\beta \ge c^*$ and folds if $\beta < c^*$.
\end{itemize}

\noindent
We conclude that the triple
\[
(a^*,b^*,c^*) 
\;\;=\;\;
\biggl(
\frac{\,a\,b\,}{\,4a^2 +5ab + b^2\,},
\;\;
\frac{\,2a^2 +4ab + b^2\,}{\,4a^2 +5ab + b^2\,},
\;\;
\frac{\,3ab + b^2\,}{\,4a^2 +5ab + b^2\,}
\biggr)
\]
solves the system and respects the ordering $0 < a^* < c^* < b^* < 1$ for all $a>0,b>0$.  


\section{Expected Value of the Game}

We now derive the \emph{ex ante} expected payoff of each player in equilibrium, i.e.\ each player's average net profit assuming they both follow these threshold strategies. 

\subsection{Case Partition in \texorpdfstring{$\alpha$}{alpha}}

Recall:
\[
\Pi_A^*(\alpha)
=\begin{cases}
\text{Bet}, & \alpha \le a^* \text{ or } \alpha \ge b^*,\\
\text{Check}, & a^*<\alpha<b^*,
\end{cases}
\quad
\Pi_B^*(\beta)
=\begin{cases}
\text{Call}, & \beta \ge c^*,\\
\text{Fold}, & \beta < c^*.
\end{cases}
\]
Since $\alpha,\beta$ are uniform(0,1), we compute Alice's expected value $\mathrm{EV}(\text{Alice})$ as
\[
\int_0^1 \bigl[\text{expected payoff given }\alpha\bigr] \,d\alpha.
\]
Break this into three regions of $\alpha$: 
\[
[\,0,\,a^*\,],\quad
[\,a^*,\,b^*\,],\quad
[\,b^*,\,1\,].
\]

\subsubsection*{1. When \texorpdfstring{$\alpha \in [0,a^*]$}{alpha in [0,a*]} (Bluff Region)}

- Alice bets.  
- Bob folds if $\beta < c^*$ (probability $c^*$), calls if $\beta \ge c^*$ (prob $1-c^*$).  
- If called, $\alpha < a^* < c^*$ implies $\beta > \alpha$ always, so Alice loses $(a+b)$.  
Thus the expected payoff for \emph{any} $\alpha$ in $[0,a^*]$ is
\[
c^*\,(+a)\;+\;(1-c^*)\,\bigl(-\,(a+b)\bigr)
\;=\;
c^*\,a \;-\;(1-c^*)\,(a+b).
\]
Because $\alpha$ is uniform, each $\alpha$ in $[0,a^*]$ has the same payoff.  So the contribution to $\mathrm{EV}(\text{Alice})$ from this region is:
\[
\int_0^{\,a^*} \Bigl[c^*\,a -(1-c^*)(a+b)\Bigr]\,d\alpha
= a^*\Bigl(c^*\,a -(1-c^*)(a+b)\Bigr).
\]

\subsubsection*{2. When \texorpdfstring{$\alpha \in [\,a^*,\,b^*\,]$}{alpha in [a*, b*]} (Check Region)}

- Alice checks.  
- No further betting occurs; they go to showdown for the pot $2a$.  
- If $\alpha > \beta$, net payoff $+a$; else $-a$.  
Hence for a given $\alpha$ in $[\,a^*,b^*\,]$, 
\[
\text{EV}(\text{Alice}\mid \alpha) = a\,(2\alpha -1).
\]
We integrate $\alpha$ from $a^*$ to $b^*$:
\[
\int_{a^*}^{\,b^*} a\,(2\alpha -1)\,d\alpha 
= a\Bigl[\alpha^2 - \alpha\Bigr]_{\,a^*}^{\,b^*}
= a\Bigl(\bigl(b^{*2}-b^*\bigr) - \bigl(a^{*2}-a^*\bigr)\Bigr).
\]

\subsubsection*{3. When \texorpdfstring{$\alpha \in [\,b^*,\,1\,]$}{alpha in [b*,1]} (Value Region)}

- Alice bets.  
- Bob folds if $\beta < c^*$ (prob $c^*$), calls if $\beta \ge c^*$ (prob $1-c^*$).  
- If called, we must consider $\alpha$ vs.\ $\beta$.  Now $\alpha \ge b^*>c^*$, so sometimes $\alpha>\beta$, sometimes $\alpha<\beta$.  
  Given $\beta\ge c^*$, $\beta$ is uniform in $[c^*,1]$, so $\Pr(\beta<\alpha)=\frac{\alpha-c^*}{1-c^*}$ and $\Pr(\beta>\alpha)=\frac{1-\alpha}{1-c^*}$.  

Hence for a \emph{fixed} $\alpha\in[b^*,1]$, the expected payoff is:
\[
c^*\,a 
+\,
(1-c^*)\Bigl[
\tfrac{\alpha-c^*}{\,1-c^*\,}\,(+\, (a+b))
\;-\;
\tfrac{1-\alpha}{\,1-c^*\,}\,(a+b)
\Bigr]
=
c^*\,a
+
(2\alpha -1 - c^*)\,(a+b).
\]
Integrate $\alpha$ from $b^*$ to $1$:
\[
\int_{\,b^*}^1 \Bigl[c^*\,a +(2\alpha -1 - c^*)(a+b)\Bigr]\;d\alpha.
\]

\subsection{Summation and Final Simplification}

Adding up these three integrals (bluff region, check region, value region) and substituting the equilibrium thresholds 
\[
a^*\;=\;\frac{\,a\,b\,}{\,4a^2 +5ab + b^2\,},\quad
b^*\;=\;\frac{\,2a^2 +4ab + b^2\,}{\,4a^2 +5ab + b^2\,},\quad
c^*\;=\;\frac{\,3ab + b^2\,}{\,4a^2 +5ab + b^2\,},
\]
one finds (after routine but somewhat lengthy algebra) that all terms cancel except for a concise final expression:
\begin{equation}
\text{EV}(\text{Alice}) 
\;=\; 
\frac{\,a^{2}\,b\,}{\,4\,a^{2} \;+\; 5\,a\,b \;+\; b^{2}\,}.
\label{Eq:AliceEV}
\end{equation}
Because this is a (two-player) zero-sum game with no rake or outside payoff, Bob's expected value is simply
\[
\text{EV}(\text{Bob})
\;=\;
-\;\text{EV}(\text{Alice})
\;=\;
-\,\frac{\,a^{2}\,b\,}{\,4\,a^{2} + 5\,a\,b + b^{2}\,}.
\]


\section{Discussion, Examples, and Conclusion}
\label{sec:DiscussionConclusion}

\subsection{Interpretation of the Formulae and Pot-Odds Balance}

From the explicit solutions
\[
\boxed{
a^* 
= \frac{\,a\,b\,}{\,4a^2 + 5ab + b^2\,}, 
\quad
b^*
= \frac{\,2a^2 + 4ab + b^2\,}{\,4a^2 + 5ab + b^2\,}, 
\quad
c^*
= \frac{\,3ab + b^2\,}{\,4a^2 +5ab + b^2\,},
}
\]
several intuitive patterns emerge:

\begin{itemize}
\item 
\emph{Bluff fraction grows with $b$}:
As the maximum bet size $b$ increases, $a^* = \tfrac{ab}{4a^2 + 5ab + b^2}$ grows, meaning Alice bluffs a larger fraction of her worst hands. A bigger bet deters Bob from calling borderline holdings, so bluffing becomes more profitable.
\item 
\emph{Value fraction also grows with $b$}:
Similarly, $b^*$ rises (away from $0.5$) as $b$ becomes larger, expanding Alice's value range. When Alice has a strong hand, a bigger bet can extract additional chips from Bob's calls.
\item 
\emph{Bob tightens his call threshold}:
The quantity $c^*$ increases in $b$, meaning Bob calls only with comparatively better $\beta$. High bet sizes demand a higher win probability to break even.
\end{itemize}

These thresholds align exactly with a pot-odds interpretation.  In equilibrium, Bob's call threshold balances the ratio of bluff to value in Alice's betting range so that his expected gain from calling is zero.  Likewise, Alice's bluff boundary $a^*$ and value boundary $b^*$ ensure that she is indifferent at each transition—thereby preventing profitable deviation from the mixed strategy.

\subsection{Example: \texorpdfstring{$(a=1,b=1)$}{(a=1, b=1)}}

A known numerical instance sets $a=1$ (so the initial pot is $2$) and $b=1$ (a pot-sized bet).  The derived thresholds become:
\[
a^*=0.1,\quad b^*=0.7,\quad c^*=0.4.
\]
Hence in equilibrium,
\[
\Pi_A^*(\alpha) 
=\begin{cases}
\text{Bet}, & \alpha < 0.1 \quad\text{(bluff)}\\
\text{Check}, & 0.1 \le \alpha \le 0.7\\
\text{Bet}, & \alpha > 0.7 \quad\text{(value)},
\end{cases}
\qquad
\Pi_B^*(\beta)
=\begin{cases}
\text{Call}, & \beta \ge 0.4,\\
\text{Fold}, & \beta < 0.4.
\end{cases}
\]
In particular, note that \(\alpha<0.1\) is $10\%$ of the worst hands (pure bluffs), and \(\alpha>0.7\) is $30\%$ of the best hands (value bets).  Meanwhile, Bob's threshold $c^*=0.4$ reflects that he must invest $1$ chip to contest a final pot of $4$ chips ($2a + 2b = 4$), implying required win probability $1/4$.  Because Alice's bluff portion is $\tfrac{a^*}{a^*+(1-b^*)} = \tfrac{0.1}{0.1 +0.3} = 0.25$, Bob's calling threshold is exactly balanced by that ratio.  

When we compute the ex ante expected value (EV) of these strategies, the bettor (Alice) gains
\[
\text{EV}(\text{Alice}) \;=\; \frac{\,1^2\cdot1\,}{\,4\cdot1^2 + 5\cdot1\cdot1 + 1^2\,}
\;=\;
\frac{1}{10}
\;=\;0.1,
\]
and Bob's EV is $-0.1$.  Thus being first to act in this one-street game yields a $+0.1$ chip advantage on average.


\subsection{Conclusion and Outlook}

In summary, this one-street von Neumann poker model captures key principles of competitive betting in a simplified setting. The equilibrium analysis reveals that Alice must adopt a mixture of high-card ``value bets'' and low-card ``bluffs,'' while Bob responds with a calling threshold that balances his pot odds against Alice's proportions of strong and weak hands. These strategies hinge on reciprocal indifferences: Alice chooses her bluff and value boundaries so that Bob's marginal call is unprofitable, and Bob chooses his call threshold so that Alice's marginal bluff (or value bet) gains no extra expectation.

Although real poker games involve multiple betting rounds, positional advantages, and further complexities (such as board cards in Hold'em), the fundamental mechanics of pot odds, threshold calling, and balancing one's betting range are clearly illustrated by this one-round model. By integrating over all possible card values, one recovers a closed-form expression for each player's expected return, showing that the aggressor (Alice) obtains a slight advantage in equilibrium. This mathematical essence, distilled into a single decision street, continues to inform more extensive multi-round poker analyses in both academic research and practical strategy development.




\begin{thebibliography}{9}

\bibitem{vNM44}
John von Neumann and Oskar Morgenstern,
\emph{Theory of Games and Economic Behavior},
Princeton University Press, 1944.

\end{thebibliography}

\end{document}